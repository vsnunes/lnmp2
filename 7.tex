%***************************************************
%* Língua Natural
%* Mini-Projeto 2 - Grupo 7
%***************************************************
\documentclass[12pt]{article}
    \usepackage[utf8]{inputenc}
    \usepackage{indentfirst}
    \usepackage{graphicx}
    \usepackage{hyperref}
    \usepackage[labelfont=bf]{caption}
    \renewcommand{\familydefault}{\sfdefault}
    \renewcommand{\refname}{Referências}
    
    \usepackage[letterpaper, portrait, margin=3cm]{geometry}
    
    \begin{document}
    \title{\vspace{-3cm}Língua Natural}
    \author{Grupo 7 - Relatório do Mini-Projeto 2}
    \date{}
    
    \maketitle
    83567 - Tiago Gonçalves
    
    83576 - Vítor Nunes
    
    \section*{Introdução}
    O projecto consiste na classifição, atribuição de categorias, de questões sobre cinema. É fornecido a um classificador um set de questões 
    com categorias previamente conhecidas e depois é possível colocar novas questões e o mesmo atribuir a categoria da mesma.  
    \section*{Proposta de Solução}
    A primeira abordagem consistiu em utilizar a biblioteca nltk \footnote{Disponível em http://www.nltk.org/} para realizar:
    \begin{itemize}
        \item{\textit{Parsing}: ler as questões conhecidas do ficheiro e criar uma estrutura de dados compatível com o classificador.}
        \item{\textit{Tokenizing}: separar as palavras da frase e, posteriormente, utilizar os lemas.}
        \item{\textit{Stemming}: ignorar palavras que aparecem em todas as frases e não são relevantes. Articuladores e pronomes, designados de \textit{stop words}.}
        \item{Naive Bayes: os classificadores utilizados foram baseados em Naive Bayes}
    \end{itemize}
    A métrica utilizada, para atribuir pesos a determinados lemas, correponde a atribuir um valor boleano (\textsc{True} ou \textsc{False}) consoante
    o novo lema já foi observado anteriormente ou não.

    Numa segunda abordagem, após realizada uma pesquisa e análise de um tutorial \cite{sklearn_tutorial}, decidiu-se utilizar:
    \begin{itemize}
        \item{\textit{Parsing} e \textit{Tokenizing}: \textsc{CountVectorizer}, que constroi uma matriz com contagens dos tokens, isto é, a frequência com que aparecem nas frases de treino.}
    \end{itemize}

    \section*{Resultados experimentais}
    A primeira abordagem revelou os seguintes dados:
    \begin{center}
        \begin{tabular}{ l | r }
          \hline
          \textbf{Classificador} & \textbf{\textit{Accuracy}} \\ \hline
          Multinomial Naive Bayes & $\approx 9.5238\%$ \\ \hline
          Bernoli Naive Bayes & $\approx2.3810\%$ \\ \hline
          Complement Naive Bayes & $\approx 9.5238\%$ \\
          \hline
        \end{tabular}
      \end{center}
      \captionof{table}{\textit{Accuracy} usando Naive Bayes sobre o ficheiro NovasQuestoes.txt \newline}

    Os resultados não foram bons, isto porque os algoritmos Naive Bayes utilizam a definição de
    probabilidade condicionada para determinar a probabilidade de dois lemas aparecerem seguidos.
    Porém o que se pretende é classificar frases em categorias específicas portanto foi necessário
    seguir outra abordagem. \newline

    A segunda abordagem, usando a biblioteca \textsc{sklearn} revelou os seguintes dados:
    \begin{center}
        \begin{tabular}{ l | l | r }
          \hline
          \textbf{Classificador} & \textbf{Kernel} & \textbf{\textit{Accuracy}} \\ \hline
          SGD\footnote{Stochastic Gradient Descent} & $hinge$ & $\approx 92,8571\%$ \\ \hline
          SGD & $log$ & $\approx 80,9524\%$ \\ \hline
          SGD & $modified_huber$ & $\approx 61,9048\%$ \\ \hline
          SGD & $squared_hinge$ & $\approx 73,8095\%$ \\ \hline
          SGD & $perceptron$ & $\approx 73,8095\%$ \\ \hline
          SGD & $epsilon\_insensitive-l2$ & $\approx 95,2381\%$ \\ \hline
          SGD & $epsilon\_insensitive-l1$ & $\approx 78,5714\%$ \\ \hline
          SGD & $epsilon\_insensitive-l2-invscalling$ & $\approx 88,0952\%$ \\ \hline
          SGD & $epsilon\_insensitive-l2-constant$ & $\approx 78,5714\%$ \\ \hline
          KNeighbors & $K = 3$ & $\approx 83,3333\%$ \\ \hline
          SVC & $linear$ & $\approx 00,0000\%$ \\ \hline
          DecisionTree & $ $ & $\approx 90,4762\%$ \\ \hline
          RandomForest & $ $ & $\approx 78,5714\%$ \\ \hline
          
          
        \end{tabular}
      \end{center}
      \captionof{table}{Comparação de vários classificadores usando o ficheiro de teste NovasQuestoes.txt \newline}
      
    O SGD ($epsilon\_insensitive$) apresentou o melhor resultado, porém após realizar mais testes, mudando
    o training set e o \textit{testing set}, obtemos melhor resultados usando o SGD ($hinge$).
    Observámos também, que o nível de falha reside em categorias que sejam bastante próximas. Por exemplo, 
    detetámos alguns erros em distinguir a categoria \textit{budget} da categoria \textit{revenue}. \newline


    \section*{Conclusão e trabalho futuro}
    Por forma a resolver erros em categorias próximas, uma possível solução seria criar uma lista de palavras
    mais comuns por categoria por forma a aperfeiçoar o critério.

    Conclui-se então, que não existe nenhum classificador pré-determinado, isto é, o classificador deve ser obtido
    à custa de testes com casos reais.


    \begin{thebibliography}{9}

        \bibitem{chatbot}
          \href{https://chatbotslife.com/text-classification-using-algorithms-e4d50dcba45}{Text Classification using Algorithms - chatbot}
        
        \bibitem{chatbot_howitworks}
        dk\_,
        \href{https://medium.com/@gk_/how-chat-bots-work-dfff656a35e2}{Soul of the Machine: How Chatbots Work}

        \bibitem{nltk_tutorial}
        PythonProgramming.net,
        \href{https://pythonprogramming.net/text-classification-nltk-tutorial/}{Text Classification with NLTK}

        \bibitem{nltk_probability}
        nltk.org
        \href{http://www.nltk.org/howto/probability.html}{Probability in NLTK}

        \bibitem{nb_class}
        Syed Sadat Nazrul,
        \href{https://towardsdatascience.com/multinomial-naive-bayes-classifier-for-text-analysis-python-8dd6825ece67}{Multinomial Naive Bayes Classifier for Text Analysis}
        
        \bibitem{nb_explained}
        Olli Huang,
        \href{https://syncedreview.com/2017/07/17/applying-multinomial-naive-bayes-to-nlp-problems-a-practical-explanation/}{Applying Multinomial Naive Bayes to NLP Problems: A Practical Explanation}

        \bibitem{sklearn}
        scikit-learn.org,
        \href{http://scikit-learn.org/stable/modules/naive_bayes.html}{1.9. Naive Bayes}

        \bibitem{sklearn_tutorial}
        scikit-learn.org,
        \href{http://scikit-learn.org/stable/tutorial/text_analytics/working_with_text_data.html}{Working With Text Data}

        \end{thebibliography}
    \end{document}
    
    